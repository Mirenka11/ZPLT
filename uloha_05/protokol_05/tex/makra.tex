% ----------------------------------------------------------------------
%  Identifikace protokolu (příkazy lze použít v celém dokumentu)
% ----------------------------------------------------------------------

%  Nastaví autora, název, datum, skupinu měření apod. 
\newcommand{\Institute}{FJFI~ČVUT~v~Praze}
%\newcommand{\Subject}{Základy fyzikálních měření}
\newcommand{\Subject}{Základní praktika z laserové techniky}  %odkomentujte dle potřeby

%  Máte-li více spoluměřících než jednoho, vložte jen jejich příjmení
\newcommand{\Author}{B -- Hrnečková, Ray}
\newcommand{\Coauthor}{Jméno kolegy} 
\newcommand{\Group}{Pátek 8:00} %den, kdy chodíte na praktika, nikoli obor
\newcommand{\Circle}{2} %číslo skupiny v rámci praktika, nikoli kruh

%  Tato část bude v každém protokolu jiná, nezapomeňte upravit!
\newcommand{\Title}{Spektrální charakteristiky optických komponentů}
\newcommand{\Labdate}{26. 2. 2025} %datum měření, nikoli datum odevzdání
\newcommand{\Worktime}{5 h} %jak dlouho vám trvalo vypracování protokolu




% ----------------------------------------------------------------------
%  Vlastní příkazy
% ----------------------------------------------------------------------


%  Matematika
\newcommand{\ee}{\mathrm{e}} %eulerovo číslo
\newcommand{\ii}{\mathrm{i}} %imaginární jednotka

% Jednotky
\renewcommand{\unit}[1]{\,\mathrm{#1}} %jednotky zadávejte pomocí tohoto příkazu
\renewcommand{\deg}{\ensuremath{\mathring{\;}}} %symbol stupně
\newcommand{\celsius}{\ensuremath{\deg\mathrm{C}}} %stupně celsia

%(hodnota plus mínus chyba) jednotka
\newcommand{\hodn}[3]{(#1 \pm #2)\unit{#3}} 

%veličina [jednotka] do hlavičky tabulky
\newcommand{\tabh}[2]{\ensuremath{#1\,[\mathrm{#2}]}} 
